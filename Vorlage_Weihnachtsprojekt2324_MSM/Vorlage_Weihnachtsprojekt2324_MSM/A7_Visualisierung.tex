\subsection*{Aufgabe 7 - Resultat und Visualisierung}

Animiert wird die Bewegung des Knickarmroboters mithilfe des \texttt{plot} Befehls. Hierfür werden die bereits in Aufgabe 1 erläuterten homogenen Transformationsmatrizen genutzt, um die generalisierten Koordinaten $\alpha$ und $\beta$ auf kartesische Koordinaten zu übersetzen. Das erste Gelenk ist hierbei gelb, das zweite grün eingefärbt. 

Die Zielposition ist mit einem orangenem Kreis markiert. 

Ein einzelnes Frame ist in Abbildung \ref{fig:vis} zu sehen. Erzeugt wird diese Animation mit dem in Listing \ref{lst:animation} dargestellten Code. 
Welche \texttt{drawnow} ausnutzt und für jedes Frame eine Pause einlegt, um die adaptive Schrittweite des Runge-Kutta solvers auszugleichen. Die Funktion \texttt{plot\_robot} erzeugt die eingefärbten Gelenke.

\begin{figure}[H]
    \centering
    \includegraphics[width=0.7\linewidth]{img/visualisierung.png}
    \caption{Visualisierung der Simulation}
    \label{fig:vis}
\end{figure}




\begin{lstlisting}[caption={Definition der rechten Seite}, label={lst:animation}]
for frame=1:n_frame
    i = floor(frame/n_frame .* numel(t));
    G2 = T_02(alphas(i), betas(i))*orig;
    G3 = T_03(alphas(i), betas(i))*orig;
    plot_robot(G2, G3)
    hold on
    plot(y_end_glob(1), y_end_glob(2), "o")
    hold off
    drawnow;
    %wait such that the animation is as long as the simulated time
    pause(time(i));
end

\end{lstlisting}


