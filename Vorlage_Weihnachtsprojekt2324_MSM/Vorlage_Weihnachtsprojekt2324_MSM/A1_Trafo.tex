\subsection*{Aufgabe 1 - Bestimmung der Kinematik und DH-Parameter}

Zur Bestimmung der Kinematik ist es notwendig herauszufinden, wie die einzelnen Gelenke des Roboters miteinander verbunden sind.
Die Denavit-Hartenberg-Notation wird verwendet, um die 3D-Transformation zum nächsten Gelenk mithilfe von vier Parameter zu beschreiben. Im Fall des Knickarmroboters ist
dies sehr einfach, da es sich lediglich um eine Kette von drei Gelenken handelt. Die Koordinatentransformation ist jeweils eine Rotation um die $z$-Achse mit $\theta_i$
und eine Translation in der $xy$-Ebene, um die Länge $a_i = l_i$. Wir können somit das erste Gelenk in den Ursprung legen und mithilfe von zwei Transformationen $T_{12}$ und $T_{23}$ alles beschreiben. Um mit einem nicht gedrehten Koordinatensystem anzufangen, benötigen wir zusätzlich $T_{01}$. Die DH-Parameter sind in Tabelle \ref{tab:DH} aufgeführt.



\begin{table}[tb]
	\centering
	\begin{tabular}{lccccr}
		\toprule
		Achse & $a_{i-1}$ & $\alpha_{i-1}$ & $d_i$ & $\theta_i$ & Art \\
		\midrule
		1 & 0& 0& 0& $-90^\circ $& Rotation \\
		2 & $l_1 = 0.16 \text{ m}$& 0& 0& $\alpha$& Translation\\
		3 & $l_2 = 0.128 \text{ m}$& 0& 0& $\beta$& Both\\ 
		\bottomrule
	\end{tabular}
	\caption{DH-Parameter des Knickarmroboters}
	\label{tab:DH}
\end{table}

Daraus resultieren die Transformationsmatrizen
\begin{equation*}
	T_{12}(\alpha, l_1) = \begin{pmatrix}
	\cos(\alpha) & -\sin(\alpha) & 0 & 0 \\
	\sin(\alpha) & \cos(\alpha) & 0 & 0 \\
	0 & 0 & 1 & 0 \\
	0 & 0 & 0 & 1
	\end{pmatrix} \cdot \begin{pmatrix}
	1 & 0 & 0 & l_1 \\
	0 & 1 & 0 & 0 \\
	0 & 0 & 1 & 0 \\
	0 & 0 & 0 & 1
	\end{pmatrix} = \begin{pmatrix}
	\cos(\alpha) & -\sin(\alpha) & 0 & \cos(\alpha) l_1 \\
	\sin(\alpha) & \cos(\alpha) & 0 & \sin(\alpha) l_1 \\
	0 & 0 & 1 & 0 \\
	0 & 0 & 0 & 1
	\end{pmatrix}
\end{equation*}
und
\begin{equation*}
	T_{23}(\beta, l_2) = \begin{pmatrix}
	\cos(\beta) & -\sin(\beta) & 0 & 0 \\
	\sin(\beta) & \cos(\beta) & 0 & 0 \\
	0 & 0 & 1 & 0 \\
	0 & 0 & 0 & 1
	\end{pmatrix} \cdot \begin{pmatrix}
	1 & 0 & 0 & l_2 \\
	0 & 1 & 0 & 0 \\
	0 & 0 & 1 & 0 \\
	0 & 0 & 0 & 1
	\end{pmatrix} = \begin{pmatrix}
	\cos(\beta) & -\sin(\beta) & 0 & \cos(\beta) l_2 \\
	\sin(\beta) & \cos(\beta) & 0 & \sin(\beta) l_2 \\
	0 & 0 & 1 & 0 \\
	0 & 0 & 0 & 1
	\end{pmatrix}
\end{equation*}

Die Gesamttransformation ergibt sich aus der Multiplikation der beiden Matrizen, welche mithilfe von trigonometrischen Additionstheoremen vereinfacht werden kann.

\begin{equation*}
	T_{13} = T_{12} \cdot T_{23} = \begin{pmatrix}
	\cos(\alpha + \beta) & -\sin(\alpha + \beta) & 0 & \cos(\alpha) l_1 + \cos(\alpha + \beta) l_2\\
	\sin(\alpha + \beta) & \cos(\alpha + \beta) & 0 & \sin(\alpha) l_1 + \sin(\alpha + \beta) l_2 \\
	0 & 0 & 1 & 0 \\
	0 & 0 & 0 & 1
	\end{pmatrix}
\end{equation*}

\begin{equation*}
	T_{03} = \begin{pmatrix}
	\cos(\alpha + \beta) & -\sin(\alpha + \beta) & 0 & \cos(\alpha) l_1 + \cos(\alpha + \beta) l_2\\
	\sin(\alpha + \beta) & \cos(\alpha + \beta) & 0 & \sin(\alpha) l_1 + \sin(\alpha + \beta) l_2 \\
	0 & 0 & 1 & 0 \\
	0 & 0 & 0 & 1
	\end{pmatrix}
\end{equation*}

Somit ergibt sich die Position von Gelenk 2 im Weltkoordinatensystem zu
\begin{equation*}
	\begin{pmatrix} x_2 \\ y_2 \\ z_2 \end{pmatrix} = \begin{pmatrix}
	\cos(\alpha) l_1\\ \sin(\alpha) l_1 \\ z_2 \end{pmatrix}
\end{equation*}
und die Position von Gelenk 3 bzw. des Endeffektors zu
\begin{equation*}
	\begin{pmatrix} x_3 \\ y_3 \\ z_3 \end{pmatrix} = 
	\begin{pmatrix}
	\cos(\alpha) l_1 + \cos(\alpha + \beta) l_2\\ \sin(\alpha) l_1 + \sin(\alpha + \beta) l_2 \\ z_3 
	\end{pmatrix}
\end{equation*}

\textit{Anmerkung}: Da die Parameter $\alpha_{i-1}$ und $d_i$ sind immer null sind, ist eine 2D Betrachtung ausreichend, wie man auch an der \emph{Identitäts-Zeile-Spalte} für $z$ erkennen kann.